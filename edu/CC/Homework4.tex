% created 19.10.2018  modified 19.10.2018

\documentclass{homework}
\begin{document}
\title{Homework 4, due date is set in Canvas LMS\\ Topic: computational chemistry of crystals with DFT}
\maketitle

\textbf{1. Lab}. Take a crystal consisting of 2-30 atoms in unit cell. Using DFT:
\begin{itemize}\setlength{\itemsep}{0ex}
\item Optimize geometry.
\item Plot pDOS.
\item Plot bands and calculate effective mass at CB minimum or VB maximum.
\item Visualize charge density distribution.
\item Calculate vibrational frequencies at Gamma point.
\item Calculate elastic tensor and its eigenvalues.
\item Estimate EoS.
\item Calculate dielectric function and UV-Vis absorption spectrum.  
\end{itemize}
The solution should be prepared in the form of a written report supplemented by the required technical files: cif-geometries, program run log-files, figures not inserted into the report etc. Be prepared to give a 5 min presentation of everything that you consider nontrivial in your work.

\bigskip\bigskip

\textit{Comments to all the exercises: 1) ``method'' means determination of both electronic structure level and basis set; 2) Briefly explain your answers.}

\bigskip

\textbf{2. Exercise}. Which method(s) you would likely use to study energy gaps in bulk semiconductors?

\bigskip



\end{document}
