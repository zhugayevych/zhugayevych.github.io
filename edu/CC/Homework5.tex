% created 26.11.2015  modified 19.10.2018

\documentclass{homework}
\begin{document}
\title{Homework 5, due date is set in Canvas LMS\\ Topic: Classical molecular dynamics of molecules and solids}
\maketitle

\textbf{1. Lab} ``Geometry optimization and molecular dynamics''. For a given molecule determine the lowest energy geometry and the room temperature dynamics using an appropriate force field. Follow the algorithm given below. Part 1 (geometry optimization):
\begin{itemize}\setlength{\itemsep}{0ex}
\item Create initial geometry. Save that geometry as xyz-file.
\item Optimize the geometry. Save the optimized geometry as xyz-file. Create a picture of the molecule. Explain the choice of the force field.
\item Compare with experiment or other method.
\item Determine and check symmetry. Determine a set of independent geometrical parameters, fundamental domain and generators.
\item Check alternative conformations. If there is a low lying metastable state, study it.
\item Explain the molecular structure.
\end{itemize}
Part 2 (molecular dynamics):
\begin{itemize}\setlength{\itemsep}{0ex}
\item Start with optimized geometry or other geometry of interest.
\item Run MD at 300 K for the minimal time to obtain a reasonably accurate sampling.
\item Save 100 snapshots to xyz-file (``movie'').
\item Determine all the conformations accessible by the MD and estimate transition rates between them.
\item Extrapolate to a laboratory time (hours).
\item Explain the observed dynamics.
\end{itemize}
The solution should be prepared in the form of a written report supplemented by the required technical files: xyz-geometries, program run log-files, figures not inserted into the report etc. Be prepared to give a 5 min presentation of everything that you consider nontrivial in your work.

\medskip

\textit{List of molecules (one per student):}
\begin{itemize}\setlength{\itemsep}{0ex}
\item any flexible molecule of your choice;
\item any min-3-units oligomer of conjugated polymers: polyyne, polyacetylene, poly(p-phenylene), poly(p-phenylene vinylene), polypyrrole, polythiophene, polyaniline, PEDOT etc.;
\item any ``multi-flexible'' part of branching alkane (e.g. isooctane) or other polymer (e.g. polybutadiene), can be cyclic;
\item any ``multi-flexible'' part of DNA, protein, or other biomolecule;
\item azobenzene, aspirine, nicotine, beta-carotene.
\end{itemize}

\bigskip

\textbf{2. Exercise} on molecular mechanics and statistical physics. Estimate the relative concentration of cis and trans conformers of butadiene molecule at the room temperature.


\end{document}
