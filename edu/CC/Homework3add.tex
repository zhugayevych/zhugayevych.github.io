% created 25.02.2014  modified 19.10.2018

\documentclass{homework}
\begin{document}
\title{Homework 3  -- additional problems}
\maketitle

\textbf{1. Exercise}. Using a case study of 3-oxopentanedial (CHO--CH$_2$--CO--CH$_2$--CHO) and formaldehyde molecules show that C=O bonds can be identified using vibrational spectroscopy by the corresponding quasilocalized IR-active C-O stretching modes. Explain the observed IR intensities and spectral positions. In the above two molecules replace C=O by C--O--H and explain the difference from the perspective of C-O stretching modes. As another counterexample show a molecule in which C=O bonds loss their ``identity'' in IR spectrum. Rationalize the obtained results: when a bond can be robustly identified through vibrational spectroscopy?

Also, for 3-oxopentanedial plot IR and Raman spectra and interpret all the prominent peaks.

\bigskip

\textbf{2. Exercise}. Using Web resources, briefly describe significance of vibrational spectroscopy of amide I, II, and III modes.

\end{document}
